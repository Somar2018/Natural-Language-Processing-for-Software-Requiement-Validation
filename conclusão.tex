\section{Conclusion and Future Work}
\sloppy
\paragraph{}
In this final section, the conclusions of the study on the use of NLP in validating software requirements are presented, with an emphasis on the benefits and limitations of NLP techniques, summarizing the main findings and making recommendations for future research, with a focus on empirical validation and in implementing automated approaches in different software development contexts.

The main findings highlighted in the state-of-the-art review on Natural Language Processing for software requirements validation emphasize the significant contribution to the automation of software requirements validation through automated analysis of specification documents. This suggests that the use of automated tools, such as those based on NLP, can improve the efficiency and accuracy of the software requirements validation process in software development projects.

Furthermore, the limitations of the natural language parser are identified, recognizing the need to rewrite sentences to be accepted and the difficulty in dealing with ambiguities. This awareness of current technology limitations highlights the importance of a realistic assessment of the capabilities and challenges faced when implementing automated software requirements validation approaches.

The need for more extensive empirical validation is highlighted as an important next step to demonstrate the effectiveness of the proposed approach. This entails carrying out case studies, controlled experiments and quantitative comparisons to validate the effectiveness, accuracy and efficiency of automated requirements analysis compared to traditional methods.

Furthermore, the importance of diversity of approaches, comparative analysis with other existing techniques and transparency and ethics in the processing of sensitive data are highlighted as fundamental aspects to be considered in the development and application of software requirements validation techniques based on NLP. These points highlight the need for a comprehensive and ethical approach when using natural language processing technologies to ensure the quality and reliability of the results obtained.

Suggested next steps in the research include conducting comprehensive empirical studies to evaluate the effectiveness, accuracy and efficiency of the proposed approach in comparison to traditional methods, generalizing the results to different types of systems and domains, discussing the practical feasibility of implementing the approach in real software development environments and the improvement of Natural Language Processing algorithms, especially in the context of validating software requirements.

These findings and next steps highlight the importance of continued research and improvement of software requirements validation techniques, aiming to improve the effectiveness, efficiency and applicability of these approaches in practical software development contexts.
