\documentclass{llncs}
%Package to generate dummy text for demonstration purposes
\usepackage{lipsum}
\usepackage{hyperref}
\usepackage{graphicx}
\usepackage{longtable} % Required for tables that span multiple pages
\usepackage{enumitem} % Required for customizing enumerate environment
\usepackage{array} % Required for customizing table columns
\usepackage{booktabs} % For better horizontal lines in tables
\usepackage{tabularx} % For auto-width columns
%\usepackage[margin=2cm]{geometry} % Adjust margin size
%\usepackage{titlesec}
%\usepackage{sectsty}
\usepackage{fancyhdr}
\usepackage{cite}
%\usepackage{microtype}
\usepackage{adjustbox}
\usepackage[english]{babel}
\usepackage[utf8]{inputenc}
\usepackage[T1]{fontenc}
\usepackage{setspace}
\setstretch{1.2} % Ajusta o espaçamento para 1.2 vezes o espaçamento padrão
\raggedbottom % Prevents LaTeX from stretching the page content to fill the height
\usepackage{babel}
\usepackage{hyphenat}
\hyphenation{your-word-list}
\usepackage{amssymb}
\tolerance=2000
\emergencystretch=12pt

% Metadata
\title{Natural Language Processing for Software Requirements Validation: State of the Art Review}
\author{António Soares Martins\inst{1} 
\and Pedro Dinis Loureiro Salgueiro\inst{2} 
\and Vítor Beires Nogueira\inst{3}}
\institute{Centro Algoritmi/LASI, University of Évora, Portugal\newline
\email{\{d54029@alunos.uevora.pt, pds@uevora.pt, vbn@uevora.pt\}}}

% Configurando o estilo de página
\pagestyle{fancy}
\fancyhf{}
\fancyhead[L]{Natural Language Processing for Software Requirements Validation}
\fancyhead[R]{\thepage}

\begin{document}
\maketitle

% Abstract
\begin{abstract}
\sloppy 
This work explores the state of the art of using Natural Language Processing (NLP) for Software Requirements Validation (SRV) in the context of software development. Requirements validation is crucial in project development, ensuring that stakeholders’ needs are understood. Without this, issues such as unnecessary or missing functionalities can arise, leading to delays, extra costs, and customer dissatisfaction. It starts with a review of the most used requirement validation methods, including traditional and modern methods that are often manual, time-consuming, and prone to human error. The aim of integrating NLP technologies is to automate and increase the accuracy of this process.
The objective of this review is to analyse and synthesise existing methodologies for validating software requirements to identify the most effective current practices in industry and academia. Furthermore, the review explores the use of new approaches to NLP, such as Large Language Models (LLMs), including BERT, GPT, BART, T5, and Llama, to enhance the validation of requirements. This work is essential for the development of an architecture that guarantees the quality and precision of the requirements and contributes to the success of software development projects.
The results obtained in this work suggest that using NLP for requirements validation can significantly reduce the time and effort required for manual reviews, improve the clarity and accuracy of the requirements, and ultimately, improve the quality and success of software projects.
\\
\textbf{Key Words}: NLP, LLMs, Software Requirements Validation, Requirements Validation TMethods
\end{abstract} 

\clearpage

% Body of the article
\section{Introduction}
\sloppy 
This study investigated how advanced Natural Language Processing (NLP) techniques can be employed to improve the validation of software requirements, detect inconsistencies, ensure completeness and integrity, ensure logical consistency, and evaluate feasibility and realism. requirements\cite{torfi2020natural}. The integration of these technologies in the validation process not only has the potential to increase the quality of the software produced, but also to mitigate risks, increase efficiency, and align requirements with the strategic objectives of organizations\cite{Wiegers2013,hafiz2016requirements}.

\subsection{Context and Importance of Software Requirements Validation}
\paragraph{}
In software development, requirement validation is essential to ensure their suitability to \textit{stakeholders}\cite{Sommerville2016}. In the context of the software development cycle, requirement validation is an initial and crucial step in which it is verified whether the captured requirements meet the needs and expectations of \textit{stakeholders} involved in the development of the system\cite{Sommerville2016 }. Challenges arise from the natural language of requirements, which are often ambiguous and inconsistent\cite{Pohl2015}. Bugs identified later in development are more expensive to fix than the design or code issues. Failure in requirements fails to meet customer needs, leading to project failures. Any change in requirements in later phases implies changes in design, architecture, or implementation. Common difficulties encountered in the software requirements process include ambiguities, incompleteness, contradictions, and frequent changes\cite{Wiegers2013}. Based on the literature\cite{bilal2016}, the main failure of software is in engineering requirements, representing 56\ % of all stages of the software development cycle. This highlights the need for significant improvements in this area.

The process of validating software requirements involves a thorough review and detailed analysis to ensure correctness, completeness, feasibility, and consistency\cite{michael2001}. Various techniques, such as reviews, inspections, prototyping, and acceptance testing, have been used to validate the requirements\cite{mcconnell2004code}. This step is important because it prevents costly errors, improves software quality, reduces risks, increases development efficiency, facilitates communication between interested parties, and ensures alignment of requirements with the organization's business objectives\cite{mcconnell2004code}.

To address these challenges effectively, innovative approaches and emerging technologies have emerged, one of which is NLP. NLP is a subarea of Artificial Intelligence and Computer Science that helps computers understand, interpret, and manipulate human languages \cite{bird2009}. The goal of NLP is to enable computers to understand the languages humans speak\cite{freeth2012nlp}. NLP techniques can be applied in various areas such as sentiment analysis, virtual assistants, \textit{chatbots}, automatic translation, text summarization, entity recognition, and recommendation systems\cite{Kumar2023}. To identify flaws in requirements and improve efficiency in their validation, we can use LLM technologies (\textit{Large Language Models}), such as BERT, GPT, LlaMA, BART, and T5\cite{Thirunavukarasu2023}. With these NLP technologies, this process can be strengthened, resulting in systems that are more aligned with the needs of the end users.

Additionally, we explore the integration of NLP technologies proposed by\cite{Alamelu2021} in the recruitment process, highlighting the significant benefits in terms of eliminating diversion and improving efficiency. Finally, the literature review of a thesis by\cite{Sayao2007} emphasizes the thesis' promising approach, which combines NLP and software agents to enhance the software requirements engineering process, aiming to optimize requirements verification and validation software efficiently and effectively.

\subsection{Objectives}
This research aims to explore the application of NLP techniques to improve the accuracy, efficiency, and effectiveness of the software requirements validation process. The main objectives include:
\begin{enumerate}
    \item \textbf{Existing Validation Methods}: Identify various software requirements validation methods and their advantages, limitations and gaps\cite{Atoum2021}.
    \item \textbf{NLP Technologies and Techniques}: Identify NLP techniques and technologies, such as the latest LLM models, that can be used for validating software requirements\cite{Gartner2024}.
   \item \textbf{Potential of Integrating NLP into software requirements validation}: Identify how to integrate NLP, such as the latest LLMs, which can capture, analyze and synthesize the validation of software requirements written in natural language, addressing ambiguities, inconsistencies, contradictions and incompleteness \cite{Gartner2024}.
\end{enumerate}



%\lipsum[1]
\clearpage

\section{Methodology}
\sloppy
\paragraph{}
The methodology adopted in this study is Literature Review, also known as State of the Art. Its purpose is to provide a comprehensive, up-to-date overview of a specific topic without the need to answer a specific research question \cite{ridley2012literature}. This approach aims to search for and synthesize recent and relevant studies, being widely used to contextualize and theoretically support new research, thus contributing to the advancement of knowledge in the area in question.

\subsection{Data sources and article selection criteria}
\begin{enumerate}
 \item \textbf{Data Sources} \\
 To perform a comprehensive and detailed review of software requirements validation, it is essential to use diverse and reliable \cite{webster2002analyzing} data sources. Key data sources include:
 \begin{itemize}
 \item {Academic Databases}: Use of renowned academic repositories such as IEEE Xplore, ACM Digital Library, SpringerLink, ScienceDirect and Google Scholar. These platforms provide access to a wide range of peer-reviewed articles, conferences, and research papers that are crucial for an in-depth understanding of practices and innovations in the field of \textbf{software} requirements validation.
 \item Books and Book Chapters: Works published by experts in the field of Software Engineering and Requirements Engineering that offer theoretical and practical insights into validating software requirements.
 \item Technical Reports and {White Papers}: Technical documents from technology companies and research institutions that discuss case studies, best practices and new methodologies.
 \item Industry Norms and Standards: Guidelines established by organizations such as IEEE, ISO and IEC, which define good practices for validating \textbf{software} requirements.\
 \end{itemize}
\item \textbf{The item selection criteria} \\
Selection of research data may vary depending on the specific objective of the study but generally involves consideration of a few key points to ensure the quality and relevance of the sources used \cite{ridley2012literature}. Here are some common criteria:
\begin{itemize}
 \item Relevance: Articles that directly address the validation of software requirements and present relevant methodologies, techniques and tools.
 \item Scientific Quality: Articles published in high-impact journals and conferences, peer-reviewed, and recognized by the academic community.
 \item Current Affairs: Preference for publications from the last ten years to ensure the inclusion of the most recent techniques and technologies.
 \item Contribution: Studies that provide significant and innovative contributions to the field, such as new methodologies, case studies, comparisons of techniques, and empirical evaluations.
 \item Diversity: Inclusion of various approaches and perspectives to obtain a complete and balanced view of the state of the art.
\end{itemize}
\item \textbf{Analysis Procedures}\\
The analysis of selected articles follows a systematic procedure to ensure the coherence and depth of the review \cite{webster2002analyzing}. The main procedures include:
\begin{itemize}
 \item Initial Reading and Screening: A first reading of the titles and abstracts of the articles is carried out to check whether they meet the inclusion criteria. Articles that are irrelevant or do not meet the criteria are discarded at this stage.
 \item Detailed Reading and Coding: For the selected articles, a detailed reading is carried out, with the coding of the main concepts, methods, results and conclusions. This involves writing down and categorizing relevant information to facilitate comparison and synthesis.
 \item Thematic Analysis: Identification of recurring themes and patterns in articles. This includes categorising validation techniques, tools used, application contexts and results obtained. Thematic analysis helps map areas of consensus and controversy in the literature.
 \item Synthesis of Results: Integration of findings into a cohesive synthesis that discusses the main methodologies for validating software requirements, their advantages, disadvantages and areas of application. The synthesis will also highlight gaps in the literature and opportunities for future research.
 \item Critical Assessment: A critical assessment of the methods and techniques discussed in the articles, considering their practical applicability, methodological robustness and limitations. This assessment is essential to determine the feasibility of applying these techniques in specific software development contexts.
\item Comparison with NLP Techniques: Analysis of how NLP techniques can complement or improve traditional software requirements validation methodologies. This comparison aims to identify synergies and propose integrated approaches for efficient validation \cite{Alamelu2021}.
\end{itemize}
\end{enumerate}
%\lipsum[2]
\clearpage

\section{Literature Review}
\sloppy
In this review, we will present the state of the art regarding software requirements validation and NLP.
\subsection{Software requirements validation methods}
Different methods are used in validating software requirements, each with specific approaches \cite{Wiegers2013}. Reviews, such as inspections, identify errors and ambiguities, prototyping helps visualize the system, modelling and simulation ensure consistency of requirements, and formal techniques offer rigour, being essential in critical systems. With technological advancements, new approaches such as NLP are integrated into software requirements validation to improve effectiveness \cite{AvilaPaldes2016, bird2009, Sommerville2016}. The correct application of these methods ensures the quality of the system, stakeholder satisfaction and the success of the project \cite{Wiegers2013}.
\begin{enumerate}
 \item \textbf{Requirements Reviews} Requirements reviews are essential to ensure the quality and adequacy of a software system's requirements. This systematic process involves a thorough analysis of requirements documents to identify potential problems such as ambiguities, inconsistencies, omissions and errors \cite{Sommerville2016}. The objectives of requirements reviews are to ensure the clarity, completeness and comprehensibility of requirements, identify and correct errors early in the development cycle, and align stakeholder's expectations with system specifications.
 To effectively apply the Requirements review method, it is recommended to bring together a diverse team of reviewers, including analysts, customers, developers and testers. Conducting informal preliminary reviews helps identify issues early in the process, while staff training ensures the consistency and productivity of reviews. Writing tests based on requirements and deriving tests from them helps ensure coverage and correspondence between requirements and tests. Mapping tests to functional requirements is critical to avoiding forgotten requirements and ensuring proper validation \cite{Sommerville2016}.
 The advantages of the requirements review method include improving software quality, reducing costs by avoiding rework, collective learning provided by the participation of a diverse team, and validating requirements against the needs of stakeholders \cite {Sommerville2016}. However, it is important to consider limitations such as the time and resources required, the possibility of bias in identifying defects, the complexity of the process, and the need for adequate training for reviewers \cite{davis2013mastering}.
 \item \textbf{Inspections and Walkthroughs}\ Model inspections and walkthroughs are software artefact review techniques used to ensure the quality, accuracy, and completeness of documents such as requirements, design, code, and documentation \cite{davis2013mastering}. \begin{itemize} \item \textbf{Inspections Model}\ The requirements inspection method is a structured and formal approach to reviewing and validating software requirements, intending to ensure that they are correct, complete, consistent and meet the needs of stakeholders. During inspections, a team of experienced reviewers examines requirements documents for errors, inconsistencies, omissions, and quality issues. The process involves planning, conducting, identifying defects, analyzing and resolving and continuous improvement. Advantages include early detection of problems, quality improvement, knowledge sharing and documentation of defects, while limitations involve time and resources, complexity, possible resistance and the need for constant updating of criteria and guidelines.
    \item \textbf{Walkthroughs}\\
 The walkthrough model is a software review technique in which the author of the document or code conducts a structured session with other team members to review and discuss the material presented. During the walkthrough process, participants identify possible problems, errors, omissions and improvements, to improve the quality of the final product. Based on \cite{Wiegers2013}, enhanced guidance is provided on how to conduct an effective requirements walkthrough. This includes preparing for a walkthrough, conducting the session, discussing and analyzing requirements, identifying improvements and post-walkthrough action and follow-up. Some advantages of this method include early identification of problems, collaborative feedback, improved understanding, and learning opportunities. However, there are limitations, such as the time and resources required, the subjectivity of the participants, the lack of focus and the need for preparation.
\end{itemize}
\item \textbf{Prototyping}
 The text highlights modelling as a systematic and mathematical approach to specifying and analyzing software requirements precisely and rigorously, using formal languages to represent system functionalities and constraints. The process of validating software requirements with modelling involves tasks such as precise specification, automated analysis, identification of conflicts, and verification of important system properties \cite{Wiegers2013}. The advantages of using modelling in validating software requirements include accuracy and clarity in representing requirements, early detection of problems, and automated verification of system properties. On the other hand, the limitations of modelling in validating software requirements involve complexity, additional cost, increased time demands, and limited interpretation for stakeholders unfamiliar with mathematical concepts.
\end{enumerate}
\subsection{NLP technologies related to LLMs}
Natural Language Processing (NLP) is a subarea of artificial intelligence focused on the interaction between computers and human language \cite{eisenstein2019}. It encompasses analysis and understanding of natural language text and speech to perform tasks such as machine translation, sentiment analysis, and named entity recognition. Using techniques from computational linguistics and machine learning, NLP enables systems to understand, interpret, and respond to textual and spoken data.

On the other hand, Large Language Models (LLMs) are NLP models trained on vast sets of textual data, containing billions of parameters, capable of predicting the next word in a sequence and capturing a wide range of linguistic patterns and contexts \cite{hou2023}. Examples of LLMs include BERT, GPT, BART, RoBERTa, ELNet, ERNIE, and T5. In addition to the individual models, there is the LLaMA collaboration (Large Language Model Meta AI), which exemplifies the evolution and capacity of these models in understanding and generating natural language. This strategic collaboration is fundamental to driving innovation in the field of language models, exploring advanced machine-learning techniques and promoting the exchange of knowledge between diverse institutions and researchers \cite{tsai2023llamaloop}.

NLP technologies with LLMs can be divided into several categories based on their usage, development and application. Key NLP technologies and techniques often interact or integrate with LLMs:

\begin{enumerate}
 \item \textbf{Text Preprocessing}\\ Text preprocessing plays a key role in the field of NLP as it prepares textual data for more advanced analysis \cite{smith2020preprocessing}.
 \begin{itemize}
    \item Tokenization: Splitting text into smaller units, such as words or subwords, to facilitate processing by LLMs.
    \item Lemmatization and Stemming: Reduction of words to their base or root form, useful for simplifying and standardizing the text.
    \item Removal of Stopwords: Elimination of common words that do not contribute significantly to the meaning of the text, reducing noise.
 \end{itemize}

 \item \textbf{Text Representation Modeling}\\
 These advanced text representation modelling techniques not only improve the accuracy of NLP applications but also open up new possibilities for NLP in diverse domains and practical applications \cite{Devlin2019}.
 \begin{itemize}
 \item Word Embeddings (e.g. Word2Vec, GloVe):\\Dense, low-dimensional representations of words in a vector space, which served as the basis for the development of LLMs \cite{acheampong2021transformer}.
 \item Contextual Embeddings (e.g., BERT, GPT, and BART): Representations that consider the context in which a word appears, allowing for a richer and more accurate understanding \cite{devlin2018bert}.
 \end{itemize}
\item\textbf{Transformer-Based Language Models}\\
 Transformer-based language models have revolutionized the field of NLP by introducing an architecture that significantly overcomes the limitations of traditional recurrent neural networks \cite{vaswani2017}
 \begin{itemize}
 \item BERT (Bidirectional Encoder Representations from Transformers): Model that captures bidirectional relationships in text, being used for tasks such as sentiment analysis, NER, and question answering \cite{devlin2018bert}.
 \item GPT (Generative Pre-trained Transformer): Model focused on text generation, used in tasks such as machine translation, text summarization and content creation \cite{Khurana2023}.
 \item T5 (\textbf{Text-To-Text Transfer Transformer}): Model that treats all NLP tasks as text-to-text conversion problems, increasing the flexibility and usefulness of the \cite{Khurana2023} model.
 \item RoBERTa (Robustly optimized BERT approach): An optimized version of BERT with better performance on various NLP tasks \cite{polignano2019alberto}.
 \item BART (Bidirectional and Auto-Regressive Transformers) represents a significant evolution in the family of transformer models, expanding the capabilities of seq2seq models by integrating both bidirectional coding and autoregressive generation into a single architecture \cite{belzner2023large}.
\end{itemize}
    These models have not only improved the accuracy of NLP tasks but also opened up new possibilities for natural language generation and understanding in a wide range of applications.
 \item \textbf{NLP Tasks Made Easy by LLMs}
 \begin{itemize}
 \item {Sentiment Analysis}: Evaluation of the sentiment expressed in texts, such as reviews of products or comments on social networks \cite{liu2022sentiment}.
 \item Named Entity Recognition (NER): Identification and classification of entities mentioned in the text, such as names of people, places, organizations \cite{belzner2023large}.
 \item Machine Translation: Automatic translation of texts between different languages.
 \item Text Summarization: Summarizing long texts into shorter, more informative versions \cite{han2021transformer}.
 \item Question Answering: Answering questions based on a set of data or texts provided \cite{Devlin2019}.
 \item Text Generation: Creating coherent and contextually relevant texts from provided prompts.
 \end{itemize}
 \item \textbf{Tools and Platforms}
 \begin{itemize}
 \item Transformers Library (Hugging Face): Popular library that makes it easy to use and implement transformer-based models \cite{hou2023}.
 \item TensorFlow and PyTorch: Machine learning frameworks that support training and implementing LLMs \cite{Beysolow2018}.
 \item SpaCy: NLP library that supports integration with pre-trained language models \cite{Beysolow2018}.
 \item NLTK (Natural Language Toolkit): Library that provides tools for basic NLP tasks, often used in conjunction with LLMs for preprocessing \cite{Beysolow2018}.
\end{itemize}
 \item \textbf{Relationship between LLMs and NLP Technologies}
\begin{enumerate}
 \item Interdependence
 \begin{itemize}
 \item LLMs rely on basic NLP techniques such as tokenization and lemmatization to preprocess text before training and inference.
 \item Advanced NLP technologies such as sentiment analysis and NER significantly benefit from LLMs' ability to understand context and generate rich representations of text.
 \end{itemize}
 \item Complementarity
 \begin{itemize}
 \item LLMs can improve the accuracy and efficiency of traditional NLP tasks by providing deeper, contextually rich understanding.
 \item NLP techniques and tools help prepare data for LLMs and refine their outputs for specific applications.
 \end{itemize}
 \item Continuous Development
 \begin{itemize}
 \item The advancement of LLMs drives the development of new NLP techniques, which in turn contribute to the evolution of even more powerful language models.
 \item Innovations in preprocessing and text representation continue to improve the performance of LLMs on a variety of NLP tasks.
 \end{itemize}

LLMs represent a significant evolution in the field of NLP, leveraging existing techniques and technologies while leveraging new capabilities and applications in NLP \cite{Devlin2019,brown2020, Liu2023,lewis2020,raffel2020t5}.
 \end{enumerate}
 \end{enumerate}
\subsubsection{}



%\lipsum[3]
\clearpage

\section{Discussion}
\paragraph{}
\sloppy
This comprehensive analysis aims to provide a holistic view of the current state of software requirements validation, highlighting where we are, existing gaps, and future opportunities for integrating advanced technologies such as NLP.
\subsection{Comparison of revised methods}
\paragraph{}
Table 1 compares these traditional and modern software requirements validation methods, highlighting the main evaluation criteria \cite{Wiegers2013, Pohl2015,freeth2012nlp, Anas2016,davis2013mastering}.
\begin{longtable}[htbp]{|p{2.5cm}|p{4.5cm}|p{4.5cm}|}
\caption{Summary Comparison of Revised Methods}\\
\hline
\textbf{Criteria} & \textbf{Traditional Methods} \newline (Reviews, Inspections, Walkthroughs) & \textbf{Modern Methods} \newline (Prototyping, Modeling) \\
\hline
\endfirsthead
%\hline \multicolumn{3}{|r|}{{Continued on next page}} \\
\hline
\endfoot
\hline
\endlastfoot
Accuracy & Good, but subject to human error & High, especially with NLP and modelling tools\\
\hline
Efficiency & Lower, due to the need for manual review & High, due to automation\\
\hline
Cost & May be high due to human time and effort & May be high due to human time and effort \\
\hline
Scalability & Difficult in large & High projects, especially with automated tools\\
\hline
Flexibility & High, but dependent on reviewers' experience & High, but requires technical knowledge of the tools\\
\hline
Complexity & Low to moderate & Moderate to high
\end{longtable}

\sloppy
Below, we will detail the criteria evaluated and how each method behaves about these criteria, based on the data in Table 1, providing a solid basis for selecting software requirements validation methods.
\begin{itemize}
\sloppy
 \item \textbf{Accuracy}: Traditional methods are effective but subject to human error, while modern methods offer high accuracy, especially with modelling. This indicates that modern methods tend to be more accurate due to automation and advanced technologies. \cite{Wiegers2013}.
 \item \textbf{Efficiency}: Traditional methods are less efficient due to the need for manual review. In contrast, modern methods are more efficient thanks to automation, which reduces the need for manual review and speeds up the validation process. \cite{Wiegers2013}.
 \item \textbf{Cost}: Both methods can have high costs due to the time and human effort involved. However, automation in modern methods can reduce costs in the long run despite the initial investment. \cite{Wiegers2013}.
 \item \textbf{Scalability}: Traditional methods struggle with large projects, while modern methods offer high scalability, especially with automated tools that can manage larger projects effectively. \cite{Wiegers2013}.
 \item \textbf{Flexibility}: Both methods are quite flexible; however, modern methods may require greater technical mastery of the tools used. \cite{Wiegers2013}.
 \item Implementation Complexity: Traditional methods have low to moderate implementation complexity. In contrast, modern methods have moderate to high implementation complexity, due to the need for technical knowledge of modern tools \cite{Wiegers2013}.
 \end{itemize}
 \sloppy
 Traditional software requirements validation methods are widely known and have been effective for many years. However, they may have limitations in terms of efficiency and scalability. On the other hand, modern methods, such as the use of NLP and modelling, provide greater accuracy and efficiency. However, its implementation requires greater technical knowledge and may involve higher initial costs. Selecting the most appropriate method should take into account the specific needs of the project, the size of the team and the availability of necessary tools and resources. \cite{Wiegers2013}.

\subsection{Identification of Gaps in Literature}
\paragraph{}
Identifying gaps in the literature is crucial to advancing scientific knowledge, highlighting areas in need of detailed research or innovative approaches. In this section, we will explore gaps in the current literature on, specific the topic, analyzing recent contributions, methodological limitations, and emerging areas that require additional attention, aiming to promote future investigations and critical dialogue about research directions.
\begin{itemize}
 \item The article "Requirements Validation Techniques Factors Influencing them" \cite{Kumar2023} highlights the broad coverage of validation techniques, the emphasis on automated tools such as NLP, machine learning and AI, and consideration of factors that influence techniques, such as system complexity and team experience.
 Recommendations to improve the selection and application of techniques include conducting comprehensive assessments, exploring automated tools, considering influencing factors, integrating techniques into the development cycle, and promoting collaboration and communication among stakeholders. Organizations must seek to constantly improve their software requirements validation practices, incorporating feedback and innovations to achieve better results and higher-quality software products.
 \item The article "Resume Validation and Filtration using Natural Language Processing" by \cite{Alamelu2021}, Addresses the use of NLP techniques to extract information from resumes. However, it lacks details on specific methods of extraction and its evaluation in terms of accuracy and effectiveness.
 It is recommended that future research explores the improvement of NLP algorithms, integration of multimodal data, customization of the screening process, large-scale performance evaluation, interpretability of results, and practical feasibility of the system in different recruitment contexts.

 \item The article "Requirements Validation via Automated Natural Language Parsing" by \cite{nanduri1995}, Highlights the innovation in using a natural language parser to validate software requirements, bringing efficiency and accuracy. Contributes to validation automation by demonstrating how automated analysis can generate object models and provide valuable feedback. The article recognizes the limitations of the parser, such as the need to rewrite sentences and deal with ambiguities, presenting a realistic view of the technology.
 It recommends more extensive empirical validation, including case studies for comparison with traditional methods, and suggests further exploring the applicability and scalability of the approach. It also proposes a detailed analysis of the practical feasibility of implementation, considering costs and integration with existing tools. For future research, it recommends comprehensive empirical studies, investigation of the generalization and scalability of the technique, integration with requirements engineering processes, use of emerging technologies, and interdisciplinary collaboration.
  \item The article MaramaAIC: tool support for consistency management and validation of requirements \cite{kamalrudin2017maramaaic} highlights an automated tool of great relevance for managing consistency and validating software requirements. The text provides a detailed description of the methodology used in the development and evaluation of the tool, highlighting its limitations, such as the need for improvements in usability and the library of interaction patterns. The generalizability of the results and future perspectives are also addressed, including the integration of advanced NLP techniques. The introduction of NLP in software requirements validation has the potential to revolutionize software development, improving the accuracy, efficiency and quality of the process. Various techniques, such as completeness analysis, consistency checking, validity validation, realism analysis, ambiguity detection and variability analysis, are explored, each presenting its challenges and benefits, highlighting the significant potential of NLP to improve quality. Of software development.
 Recommendations include layout and labelling improvements, integration with GUI templates, improving traceability support, supporting partial selection of changes, expanding the library of interaction patterns, and conducting a longitudinal study. These suggestions aim to optimize the user experience and ensure consistency and effectiveness in the validation and traceability of requirements in software development projects.

 \item The article "Challenges of Software Requirements Quality "Assurance and Validation: A Systematic Literature Review", presents a systematic review of the literature on the validation of software requirements \cite{Atoum2021}. Analyze the most adopted validation techniques, the quality characteristics of the requirements the significant challenges faced in this process categories of quality characteristics of the requirements, and tools and datasets adopted in these techniques. Validation techniques were grouped into categories such as prototyping. , inspection, knowledge-oriented, test-oriented, modelling and evaluation, and formal models highlight the trend of applying machine learning techniques in validating software requirements, along with challenges related to requirements expression and customer feedback. Some important recommendations include exploring different domains: It is recommended to conduct studies and tests in a variety of domains to evaluate the effectiveness and applicability of software requirements validation techniques in different contexts. Scalability: Consider the scalability of software requirements validation techniques to ensure they can handle systems of different sizes and complexities. Comparison with other methodologies: Carry out comparisons between different software requirements validation methodologies to identify the advantages and disadvantages of each approach and determine the most appropriate one for a given scenario. Model interpretability techniques: Investigate and develop techniques that improve the interpretability of models used in validating software requirements, to facilitate the understanding of decisions made by the system.
 \end{itemize}

Table 2 summarizes the gaps and recommendations identified by the reviewed articles.
\begin{longtable}[htbp]{|p{2.70cm}|p{4.5cm}|p{4.5cm}|}
\caption{Summary of Gaps and Recommendations}\\
\hline
\textbf{Category} & \textbf{Identified Gaps} & \textbf{Recommendations} \\
\hline
\endfirsthead
\hline
\endhead
\hline
\endfoot
\hline
\endlastfoot
\sloppy
Evaluation and Comparison of Techniques \cite{kumar2021, Alamelu2021, nanduri1995, Atoum2021}
&
\raggedright 1. Lack of clear performance metrics and comprehensive comparisons with other approaches \newline.
2. Lack of need for detailed empirical studies to validate techniques.
 &
\raggedright
1. Develop clear performance metrics and perform comprehensive comparisons with other approaches.\newline
2. Carry out detailed empirical studies to validate the proposed techniques.\\
\cr
\hline
\raggedright
Exploration and Integration of Tools \cite{kumar2021,Alamelu2021,nanduri1995} &
1. There is a lack of need for continuous improvements in NLP algorithms.\newline
2. Lack of multimodal data integration.\newline
3. Lack of in-depth exploration of automated tools in different contexts and domains.
 &
1. Continuous improvements to NLP algorithms.\newline
2. It is necessary to integrate multimodal data.
3. Explore more deeply the use of automated tools in different contexts and domains.\\

\hline
\sloppy Influencing and Contextual Factors \cite{kumar2021,Atoum2021} &
1. Lack of detailed consideration of how different contextual factors influence the effectiveness of techniques\newline.
2. There is a lack of variety of domains to evaluate the generalization of techniques.
 &
1. Consider factors such as system complexity and team experience.
2. Explore different domains to evaluate the applicability of techniques.\\

\hline
Scalability \&\ Generalization \cite{Atoum2021,nanduri1995}
&
1. Lack of need for scalability of techniques in large and complex systems.\newline
2. Lack of validation of the applicability of techniques in different scenarios and project sizes.
&
1. Improving the scalability of techniques for large and complex systems.\newline
2. Validation of the applicability of techniques in different scenarios and project dimensions.\\

\hline
 Integration in the Development Cycle,\cite{kumar2021,nanduri1995} &
1. Lack of clear guidelines on how to effectively integrate validation techniques into the existing development cycle.\newline
2. There is a lack of need for studies on the effectiveness of integration at different phases of the development cycle.
\newline
 &
1. Development of clear guidelines for the effective integration of validation techniques into the existing development cycle.\newline
2. Carrying out studies on the effectiveness of integration at different stages of the development cycle. \\
\hline

Improved collaboration and communication between stakeholders \cite{kumar2021}
&
1. Lack of studies on best practices to promote collaboration and effective communication between stakeholders during software requirements validation.\newline
2. There is a lack of need to explore the impact of communication on the success of software requirements validation.
&
Conduct systematic studies to identify and document best practices that promote collaboration and effective communication between \textit{stakeholders} during software requirements validation.

Investigate the impact of communication on the effectiveness of requirements validation, aiming to optimize the project's final results.\\

\hline
\raggedright
Interpretation and Understanding of Models
\cite{Atoum2021, nanduri1995, Alamelu2021}
&
\raggedright
1. Lack of effective techniques to improve model interpretability.\newline
2. Need for comprehensive empirical studies to validate interpretability.\newline
3. Lack of adequate customization of the screening process for different contexts.

 &
\raggedright
1. Continuous research and development of innovative techniques to improve the interpretability of the models used.\newline
2. Conducting comprehensive empirical studies to validate and compare the interpretability of these models.\newline
3. Adaptation and customization of the screening process to meet the specific needs of different contexts and application scenarios.

\end{longtable}

\subsection{Potentials Integration of NLP in Software Requirements Validation}
The integration of NLP technologies into software requirements validation has the potential to revolutionize software development, providing advances in accuracy, efficiency and quality.\cite{Alamelu2021}. These techniques can be applied in several areas, each with its benefits and challenges \cite{Wiegers2013}. For example:
\begin{itemize}
 \item Completeness and integrity analysis ensure the presence of all necessary aspects in the requirements, using tokenization, embeddings and automatic summarization to identify gaps and reduce manual review time \cite{Acheampong2021}. However, this approach requires specific training and structured knowledge bases. \cite{polignano2019alberto}.
 \item Consistency checking ensures the logical coherence of requirements through techniques such as cosine similarity and named entity recognition, although it is challenging to set appropriate thresholds and deal with multiple valid interpretations \cite{Kamalrudin2015, Belzner2023}.
 \item Validity validation verifies that requirements correctly reflect stakeholder expectations, using sentiment analysis and text classification to improve clarity and alignment, despite difficulties in capturing emotional nuances \cite{lewis2020,sonbol2022,polignano2019alberto}.
 \item Realism analysis evaluates the feasibility of requirements by comparing them with existing benchmarks, facing challenges in maintaining updated knowledge bases\cite{Devlin2018}.
 \item Ambiguity detection identifies ambiguous terms and suggests fixes to improve documentation, although dealing with variable contexts is complex \cite{Devlin2018}.
 \item Variability analysis considers the flexibility of requirements, identifying patterns and modelling rules, but requires capturing a wide range of scenarios \cite{Devlin2018}.
\end{itemize}
In summary, the integration of NLP techniques into software requirements validation presents significant potential for improving the quality of software development, despite the associated challenges.\newline
The article \cite{Gartner2024} introduces the ALICE platform, which combines formal logic and large language models (LLMs) to detect contradictions in engineering requirements using precise reasoning, rule-based inference, and pattern recognition from LLMs. Its methodology, with expanded taxonomy and decision tree model, demonstrates superiority in accuracy, recall and precision about methods based only on LLMs.\newline.
It is recommended that future research explore different domains, scalability, comparisons with other methodologies and model interpretability techniques.
\newline
The article \cite{kamalrudin2017maramaaic} presents the MaramaAIC (Automated Inconsistency Checker) tool (Automated Inconsistency Checker) for consistency management and validation of software requirements, with automated traceability and visual support. This tool benefits the requirements engineering community and software development professionals. It stands out for its clarity in methodology and evaluation, but it is necessary to consider the scope and representativeness of the studies. Limitations include improvements in usability and expansion of the library of interaction patterns, with recommendations for future research to integrate advanced Natural Language Processing techniques.

%\lipsum[4]
\clearpage

\section{Conclusion and Future Work}
\sloppy
\paragraph{}
In this final section, the conclusions of the study on the use of NLP in validating software requirements are presented, with an emphasis on the benefits and limitations of NLP techniques, summarizing the main findings and making recommendations for future research, with a focus on empirical validation and in implementing automated approaches in different software development contexts.

The main findings highlighted in the state-of-the-art review on Natural Language Processing for software requirements validation emphasize the significant contribution to the automation of software requirements validation through automated analysis of specification documents. This suggests that the use of automated tools, such as those based on NLP, can improve the efficiency and accuracy of the software requirements validation process in software development projects.

Furthermore, the limitations of the natural language parser are identified, recognizing the need to rewrite sentences to be accepted and the difficulty in dealing with ambiguities. This awareness of current technology limitations highlights the importance of a realistic assessment of the capabilities and challenges faced when implementing automated software requirements validation approaches.

The need for more extensive empirical validation is highlighted as an important next step to demonstrate the effectiveness of the proposed approach. This entails carrying out case studies, controlled experiments and quantitative comparisons to validate the effectiveness, accuracy and efficiency of automated requirements analysis compared to traditional methods.

Furthermore, the importance of diversity of approaches, comparative analysis with other existing techniques and transparency and ethics in the processing of sensitive data are highlighted as fundamental aspects to be considered in the development and application of software requirements validation techniques based on NLP. These points highlight the need for a comprehensive and ethical approach when using natural language processing technologies to ensure the quality and reliability of the results obtained.

Suggested next steps in the research include conducting comprehensive empirical studies to evaluate the effectiveness, accuracy and efficiency of the proposed approach in comparison to traditional methods, generalizing the results to different types of systems and domains, discussing the practical feasibility of implementing the approach in real software development environments and the improvement of Natural Language Processing algorithms, especially in the context of validating software requirements.

These findings and next steps highlight the importance of continued research and improvement of software requirements validation techniques, aiming to improve the effectiveness, efficiency and applicability of these approaches in practical software development contexts.

%\lipsum[5]
\clearpage

% Bibliography
\bibliographystyle{splncs04}
\bibliography{bibliografy} % Specify your .bib file
\end{document}
