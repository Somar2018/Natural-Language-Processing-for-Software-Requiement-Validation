\section{Introduction}
\sloppy 
This study investigated how advanced Natural Language Processing (NLP) techniques can be employed to improve the validation of software requirements, detect inconsistencies, ensure completeness and integrity, ensure logical consistency, and evaluate feasibility and realism. requirements\cite{torfi2020natural}. The integration of these technologies in the validation process not only has the potential to increase the quality of the software produced, but also to mitigate risks, increase efficiency, and align requirements with the strategic objectives of organizations\cite{Wiegers2013,hafiz2016requirements}.

\subsection{Context and Importance of Software Requirements Validation}
\paragraph{}
In software development, requirement validation is essential to ensure their suitability to \textit{stakeholders}\cite{Sommerville2016}. In the context of the software development cycle, requirement validation is an initial and crucial step in which it is verified whether the captured requirements meet the needs and expectations of \textit{stakeholders} involved in the development of the system\cite{Sommerville2016 }. Challenges arise from the natural language of requirements, which are often ambiguous and inconsistent\cite{Pohl2015}. Bugs identified later in development are more expensive to fix than the design or code issues. Failure in requirements fails to meet customer needs, leading to project failures. Any change in requirements in later phases implies changes in design, architecture, or implementation. Common difficulties encountered in the software requirements process include ambiguities, incompleteness, contradictions, and frequent changes\cite{Wiegers2013}. Based on the literature\cite{bilal2016}, the main failure of software is in engineering requirements, representing 56\ % of all stages of the software development cycle. This highlights the need for significant improvements in this area.

The process of validating software requirements involves a thorough review and detailed analysis to ensure correctness, completeness, feasibility, and consistency\cite{michael2001}. Various techniques, such as reviews, inspections, prototyping, and acceptance testing, have been used to validate the requirements\cite{mcconnell2004code}. This step is important because it prevents costly errors, improves software quality, reduces risks, increases development efficiency, facilitates communication between interested parties, and ensures alignment of requirements with the organization's business objectives\cite{mcconnell2004code}.

To address these challenges effectively, innovative approaches and emerging technologies have emerged, one of which is NLP. NLP is a subarea of Artificial Intelligence and Computer Science that helps computers understand, interpret, and manipulate human languages \cite{bird2009}. The goal of NLP is to enable computers to understand the languages humans speak\cite{freeth2012nlp}. NLP techniques can be applied in various areas such as sentiment analysis, virtual assistants, \textit{chatbots}, automatic translation, text summarization, entity recognition, and recommendation systems\cite{Kumar2023}. To identify flaws in requirements and improve efficiency in their validation, we can use LLM technologies (\textit{Large Language Models}), such as BERT, GPT, LlaMA, BART, and T5\cite{Thirunavukarasu2023}. With these NLP technologies, this process can be strengthened, resulting in systems that are more aligned with the needs of the end users.

Additionally, we explore the integration of NLP technologies proposed by\cite{Alamelu2021} in the recruitment process, highlighting the significant benefits in terms of eliminating diversion and improving efficiency. Finally, the literature review of a thesis by\cite{Sayao2007} emphasizes the thesis' promising approach, which combines NLP and software agents to enhance the software requirements engineering process, aiming to optimize requirements verification and validation software efficiently and effectively.

\subsection{Objectives}
This research aims to explore the application of NLP techniques to improve the accuracy, efficiency, and effectiveness of the software requirements validation process. The main objectives include:
\begin{enumerate}
    \item \textbf{Existing Validation Methods}: Identify various software requirements validation methods and their advantages, limitations and gaps\cite{Atoum2021}.
    \item \textbf{NLP Technologies and Techniques}: Identify NLP techniques and technologies, such as the latest LLM models, that can be used for validating software requirements\cite{Gartner2024}.
   \item \textbf{Potential of Integrating NLP into software requirements validation}: Identify how to integrate NLP, such as the latest LLMs, which can capture, analyze and synthesize the validation of software requirements written in natural language, addressing ambiguities, inconsistencies, contradictions and incompleteness \cite{Gartner2024}.
\end{enumerate}


