\section{Methodology}
\sloppy
\paragraph{}
The methodology adopted in this study is Literature Review, also known as State of the Art. Its purpose is to provide a comprehensive, up-to-date overview of a specific topic without the need to answer a specific research question \cite{ridley2012literature}. This approach aims to search for and synthesize recent and relevant studies, being widely used to contextualize and theoretically support new research, thus contributing to the advancement of knowledge in the area in question.

\subsection{Data sources and article selection criteria}
\begin{enumerate}
 \item \textbf{Data Sources} \\
 To perform a comprehensive and detailed review of software requirements validation, it is essential to use diverse and reliable \cite{webster2002analyzing} data sources. Key data sources include:
 \begin{itemize}
 \item {Academic Databases}: Use of renowned academic repositories such as IEEE Xplore, ACM Digital Library, SpringerLink, ScienceDirect and Google Scholar. These platforms provide access to a wide range of peer-reviewed articles, conferences, and research papers that are crucial for an in-depth understanding of practices and innovations in the field of \textbf{software} requirements validation.
 \item Books and Book Chapters: Works published by experts in the field of Software Engineering and Requirements Engineering that offer theoretical and practical insights into validating software requirements.
 \item Technical Reports and {White Papers}: Technical documents from technology companies and research institutions that discuss case studies, best practices and new methodologies.
 \item Industry Norms and Standards: Guidelines established by organizations such as IEEE, ISO and IEC, which define good practices for validating \textbf{software} requirements.\
 \end{itemize}
\item \textbf{The item selection criteria} \\
Selection of research data may vary depending on the specific objective of the study but generally involves consideration of a few key points to ensure the quality and relevance of the sources used \cite{ridley2012literature}. Here are some common criteria:
\begin{itemize}
 \item Relevance: Articles that directly address the validation of software requirements and present relevant methodologies, techniques and tools.
 \item Scientific Quality: Articles published in high-impact journals and conferences, peer-reviewed, and recognized by the academic community.
 \item Current Affairs: Preference for publications from the last ten years to ensure the inclusion of the most recent techniques and technologies.
 \item Contribution: Studies that provide significant and innovative contributions to the field, such as new methodologies, case studies, comparisons of techniques, and empirical evaluations.
 \item Diversity: Inclusion of various approaches and perspectives to obtain a complete and balanced view of the state of the art.
\end{itemize}
\item \textbf{Analysis Procedures}\\
The analysis of selected articles follows a systematic procedure to ensure the coherence and depth of the review \cite{webster2002analyzing}. The main procedures include:
\begin{itemize}
 \item Initial Reading and Screening: A first reading of the titles and abstracts of the articles is carried out to check whether they meet the inclusion criteria. Articles that are irrelevant or do not meet the criteria are discarded at this stage.
 \item Detailed Reading and Coding: For the selected articles, a detailed reading is carried out, with the coding of the main concepts, methods, results and conclusions. This involves writing down and categorizing relevant information to facilitate comparison and synthesis.
 \item Thematic Analysis: Identification of recurring themes and patterns in articles. This includes categorising validation techniques, tools used, application contexts and results obtained. Thematic analysis helps map areas of consensus and controversy in the literature.
 \item Synthesis of Results: Integration of findings into a cohesive synthesis that discusses the main methodologies for validating software requirements, their advantages, disadvantages and areas of application. The synthesis will also highlight gaps in the literature and opportunities for future research.
 \item Critical Assessment: A critical assessment of the methods and techniques discussed in the articles, considering their practical applicability, methodological robustness and limitations. This assessment is essential to determine the feasibility of applying these techniques in specific software development contexts.
\item Comparison with NLP Techniques: Analysis of how NLP techniques can complement or improve traditional software requirements validation methodologies. This comparison aims to identify synergies and propose integrated approaches for efficient validation \cite{Alamelu2021}.
\end{itemize}
\end{enumerate}