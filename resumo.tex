\begin{abstract}
\sloppy 
This work explores the state of the art of using Natural Language Processing (NLP) for Software Requirements Validation (SRV) in the context of software development. Requirements validation is crucial in project development, ensuring that stakeholders’ needs are understood. Without this, issues such as unnecessary or missing functionalities can arise, leading to delays, extra costs, and customer dissatisfaction. It starts with a review of the most used requirement validation methods, including traditional and modern methods that are often manual, time-consuming, and prone to human error. The aim of integrating NLP technologies is to automate and increase the accuracy of this process.
The objective of this review is to analyse and synthesise existing methodologies for validating software requirements to identify the most effective current practices in industry and academia. Furthermore, the review explores the use of new approaches to NLP, such as Large Language Models (LLMs), including BERT, GPT, BART, T5, and Llama, to enhance the validation of requirements. This work is essential for the development of an architecture that guarantees the quality and precision of the requirements and contributes to the success of software development projects.
The results obtained in this work suggest that using NLP for requirements validation can significantly reduce the time and effort required for manual reviews, improve the clarity and accuracy of the requirements, and ultimately, improve the quality and success of software projects.
\\
\textbf{Key Words}: NLP, LLMs, Software Requirements Validation, Requirements Validation TMethods
\end{abstract} 
